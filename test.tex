% !TeX encoding = UTF-8
% !TeX program = XeTeX
\documentclass{article}
\usepackage{xeCJK} % 须放在\usepackage{}列中足够前的位置
\usepackage{fontspec}
\usepackage{lipsum}
\usepackage{fancyhdr}
\usepackage[square,sort,comma,numbers]{natbib}


% 在导言区定义abstract环境
\newenvironment{myabstract}{%
% 开始时执行
  \small
  \bfseries Abstarct:\ %
}{%
  \par
}

% 在导言区定义关键词环境
\newenvironment{keywords}{%
  \small
  \bfseries Keywords:\ %
}{%
  \par
}


%----------页眉页脚----------------------%
% 设置除首页外的页面样式
\fancypagestyle{normal}{
    \fancyhf{} % 清空页眉页脚
    \fancyhead[L]{中国科学院大学人工智能学院}
    \fancyhead[R]{深度学习报告}
    \fancyfoot[R]{\thepage}
    \renewcommand{\headrulewidth}{0.4pt} % 页眉分隔线
    \renewcommand{\footrulewidth}{0.4pt} % 页脚分隔线
}
% 设置文档的默认页面样式为 normal
\pagestyle{normal}
%----------页眉页脚----------------------%

%文章区
\begin{document}

%----------标题----------------------%
\title{Deep Learning Report}
\author{张豪派}
\date{2024年5月}
\maketitle
\thispagestyle{fancy} % 将标题页样式设置为 fancy
%----------页眉页脚----------------------%


%----------abstract & keywords----------------------%
\begin{myabstract}
    \lipsum[1]
\end{myabstract}
\begin{keywords}
    关键词1, 关键词2, 关键词3
\end{keywords}


%----------abstract & keywords----------------------%


%----------正文----------------------%
\section{问题陈述}
\lipsum[1-5] % Just some dummy text to fill the pages


\section{研究现状与相关工作}
\lipsum[1-3]

\section{实验复现与改进}

这是一个引用示例 \cite{pang2021deep} \cite{5374676}。

\section{结论}
\lipsum[1-3]



\bibliography{MyReference}
\bibliographystyle{IEEEtran}



%----------正文----------------------%
\end{document}

